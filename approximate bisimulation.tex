\documentclass[contract,10pt]{elsarticle}

\usepackage{float}
\usepackage[utf8]{inputenc}
\usepackage[T1]{fontenc}
\usepackage{amsmath,amssymb,stmaryrd,txfonts,dsfont}
\usepackage{amsbsy}
\usepackage{lmodern}
\usepackage{graphicx}
\usepackage{url}
\usepackage{array}
\usepackage{hyperref}
\usepackage[all]{xy}
\usepackage{tikz,tkz-graph}
\usetikzlibrary{arrows,automata,positioning}
\usetikzlibrary{trees,shapes,calc}
%\usepackage{wrapfig,subfigure}
%\usepackage{gastex}
%\usepackage[strings]{underscore} % for correct handling of underscore in DOIs
%\usepackage{undertilde}
\usepackage{breakurl}
\usepackage[ruled,vlined]{algorithm2e}

%\setcounter{tocdepth}{3}
%\newcommand{\keywords}[1]{\par\addvspace\baselineskip
%\noindent\keywordname\enspace\ignorespaces#1}
%\newcommand{\mpar}[1]{\marginpar{#1}}
%\newcommand{\qed}{$\square$}
%=======================================================================
\newtheorem{defi}{Definition}
\newtheorem{remark}{Remark}
\newtheorem{theo}{Theorem}
\newtheorem{prop}{Proposition}
\newtheorem{lemm}{Lemma}
\newtheorem{coro}{Corollary}
\newtheorem{exam}{Example}
%
\newenvironment{definition}{\begin{defi} \rm }{\end{defi}}
\newenvironment{theorem}{\begin{theo} \rm }{\end{theo}}
\newenvironment{proposition}{\begin{prop} \rm }{\end{prop}}
\newenvironment{lemma}{\begin{lemm} \rm }{\end{lemm}}
\newenvironment{corollary}{\begin{coro} \rm }{\end{coro}}
\newenvironment{example}{\begin{exam} \rm }{\end{exam}}
\newenvironment{proof}{{\noindent \bfseries Proof.}}{\qed}


%====================================================

%\newcommand{\pf}{\noindent {\bf Proof. }}
\newcommand{\leaveout}[1]{}

\newcommand{\cl}{R^{*}}
\newcommand{\ds}[1]{\lbrack \hspace{-0.6mm}\lbrack #1 \rbrack \hspace{-0.6mm}\rbrack}
\newcommand{\ls}[1]{|\!|#1 |\!|}
\newcommand{\mint}[2]{\int\!#1\,d #2}
\newcommand{\com}{\bullet\!,}
\newcommand{\vc}{\mbox{$\ \bigcirc\hspace{-3.5mm}\vee\ $}}
\newcommand{\wc}{\mbox{$\ \bigcirc\hspace{-3.5mm}\wedge\ $}}
\newcommand{\lift}[1]{\mathrel{{#1}^\dag}}
\newcommand{\liftcl}[1]{\mathrel{{#1}^\circ}}

%=================================
\newcommand{\taolue}[1]{\color{green} {TL: #1 :LT} \color{black}}
\newcommand{\tingting}[1]{{\color{red}TT: #1 :TT}}
\newcommand{\hengyang}[1]{{\color{red}HY: #1 :HY}}
\newcommand{\Y}[1]{{\color{blue}   #1}}
\newcommand{\R}[1]{{\color{red}   #1}}


%=================================


%\def\br{\begin{re}}
%	\def\bd{\begin{de}}
%		\def\bt{\begin{th1}}
%			\def\bl{\begin{lemma}}
%				\def\bp{\begin{ppro}}
%					\def\bx{\begin{exe}}
%						\def\btl{\begin{tuilun}}
%							\def\bc{\begin{center}}
%								\def\bco{\begin{co}}
%									\def\ed{\end{de}}
%								\def\et{\end{th1}}
%							\def\ep{\end{ppro}}
%						\def\el{\end{lemma}}
%					\def\ex{\end{exe}}
%				\def\ec{\end{center}}
%			\def\bi{\begin{itemize}}
%				\def\ei{\end{itemize}}
%			\def\eco{\end{co}}
%		\def\er{\end{re}}
	% MATH -----------------------------------------------------------
	\newcommand{\norm}[1]{\left\Vert#1\right\Vert}
	\newcommand{\abs}[1]{\left\vert#1\right\vert}
%	\newcommand{\set}[1]{\left\{#1\right\}}
	\newcommand{\Real}{\mathbb R}
	\newcommand{\eps}{\varepsilon}
%	\newcommand{\To}{\longrightarrow}
	\newcommand{\BX}{\mathbf{B}(X)}
	\newcommand{\A}{\mathcal{A}}
	
	% Sponsored Auction ----------------------------------------------
	\newcommand{\bids}{\mathbf{b}}
	\newcommand{\valueonpos}[1]{v_{#1}}
	\newcommand{\bidonpos}[1]{b_{#1}}
	\newcommand{\utility}[2]{u^{#1}_{#2}}
	\newcommand{\dom}[1]{\text{dom}{#1}}
	\newcommand{\ran}[1]{\text{ran}{#1}}
	%\def\ez{\hfill $\Box$ \vspace{2mm}}
	\newcommand{\nint}[2]{\int\!#1\,d #2}
	\newcommand{\FB}{\mathit{FB}}
	
%====================================================================

\begin{document}


\begin{frontmatter}
\title{Compositionality Metaresults of $\epsilon$-Bisimulations for Fuzzy Transition Systems} %\tnoteref{fund}}
%\tnotetext[fund]{}
\end{frontmatter}


\begin{definition}(bisimulation) \label{bisimulation} Let $\cal {M}=(S,A,\rightarrow)$ be an FTS.

\begin{itemize}

\item A \R {symmetric} relation $R$ over $S$ is a \textit{pre}-bisimulation iff, whenever $(s,t)\in R$, then for all actions $a\in
A$ and \underline{all $R$-closed sets $U$} it holds that for each $s\xrightarrow{a}\mu$ there exists
$t\xrightarrow{a}\nu$ such that $\mu(U)=\nu(U)$. Two states $s$ and $t$ are pre-bisimilar, denoted as $s\sim^{pre} t$, if there is a pre-bisimulation $R$ that relates them.


\item  A \R {symmetric} relation $R$ over $S$ is a \textit{post}-bisimulation iff, whenever $(s,t)\in R$, then for all actions $a\in
A$ it holds that for each $s\xrightarrow{a}\mu$ there exists
$t\xrightarrow{a}\nu$ such that $ \mu(U)=\nu(U)$ for \underline {all $R$-closed sets $U$}. Two states $s$ and $t$ are post-bisimilar, denoted as $s\sim^{post} t$, if there is a post-bisimulation $R$ that relates them.

\end{itemize}

\end{definition}

\begin{definition} ($\epsilon$-Bisimulation)\label{approximate bisimulation}
 Let $\cal {M}=(S,A,\rightarrow)$ be an FTS and $\epsilon\geq 0$.

 \begin{itemize}

\item A \R {symmetric}
relation $R$ over $S$ is an $\epsilon$-pre-bisimulation iff, whenever $(s,t)\in R$, then for all actions $a\in
A$ and \underline{all $R$-closed sets $U$} it holds that for each $s\xrightarrow{a}\mu$ there exists
$t\xrightarrow{a}\nu$ such that $$|\mu(U)-\nu(U)|\leq \epsilon.$$ Two states  $s$ and $t$ are $\epsilon$-pre-bisimilar, denoted as $s\sim^{pre}_{\epsilon} t$, if there is an $\epsilon$-pre-bisimulation $R$ that relates them.

\item A \R {symmetric}
relation $R$ over $S$ is an $\epsilon$-post-bisimulation iff, whenever $(s,t)\in R$, then for all actions $a\in
A$ and it holds that for each $s\xrightarrow{a}\mu$ there exists
$t\xrightarrow{a}\nu$ such that $$|\mu(U)-\nu(U)|\leq \epsilon$$ for \underline {all $R$-closed sets $U$}. Two states  $s$ and $t$ are $\epsilon$-post-bisimilar, denoted as $s\sim^{post}_{\epsilon} t$, if there is an $\epsilon$-post-bisimulation $R$ that relates them.

\end{itemize}
\end{definition}

For convenience, we appoint: when some properties  both hold for $\sim^{pre}$ and $\sim^{post}$, we use the symbol $\sim $; and some properties both hold for $\sim^{pre}_{\epsilon}$ and $\sim^{post}_{\epsilon}$, we use the symbol
$\sim_{\epsilon}$.

Clearly, $s\sim t$ iff $s\sim_{0} t$.

\begin{proposition}  If $R$ is  an $\epsilon$-post-bisimulation, then $R$ is  an $\epsilon$-pre-bisimulation. Further, $\sim^{post}_{\epsilon}\subseteq \sim^{pre}_{\epsilon}$ for any $\epsilon$.
\end{proposition}

\begin{proposition}\label{property}

 (1) $\sim_{\epsilon_1}\subseteq \sim_{\epsilon_2} $ whenever $\epsilon_1\leq \epsilon_2$;

 (2)$s\sim_{\epsilon_1} s'$ and $s'\sim_{\epsilon_2} t$ then $s\sim_{\epsilon_1+\epsilon_2} t$;

 (3)if $R_i$ is an $\epsilon_i$-bisimulation ($i=1,2$), then $R_1\circ R_2$ is $(\epsilon_1+\epsilon_2)$-bisimulation.

\end{proposition}

\begin{proposition}  $\sim_{\epsilon}$ is reflexive and symmetric, moreover it is the greatest $\epsilon$-bisimulation, that is
$$
\sim_{\epsilon}=\cup\{R\mid \mbox{$R$ is an $\epsilon$-bisimulation}\}
$$
\end{proposition}

\R {Note that $\sim_{\epsilon}$ is not necessarily transive, i.e., it is not an equivalence relation. }


Define:

(1) $F^{pre}_{\epsilon}:2^{S\times S}\rightarrow 2^{S\times S}$ as follows:
 \label{eq:defF}
$$F^{pre}_{\epsilon}(R)=\left\{(s,t)\in S\times S \put(2,-11){\line(0,1){25.0}}
\hspace{-0.2cm}\begin{array}{lll} &\mbox{for any $R$-closed set $U$}&\\
&\forall s\xrightarrow{a} \mu,~\exists t\xrightarrow{a}\nu:~ |\mu(U)-\nu(U)|\leq \epsilon&\\
&\forall t\xrightarrow{a} \nu,~\exists s\xrightarrow{a}\mu:~ |\mu(U)-\nu(U)|\leq \epsilon&\\
\end{array}
\right\}
$$

(2)$F^{post}_{\epsilon}:2^{S\times S}\rightarrow 2^{S\times S}$ as follows:
 \label{eq:defF}
$$F^{post}_{\epsilon}(R)=\left\{(s,t)\in S\times S \put(2,-11){\line(0,1){25.0}}
\hspace{-0.2cm}\begin{array}{lll}
&\forall s\xrightarrow{a} \mu,~\exists t\xrightarrow{a}\nu:~ |\mu(U)-\nu(U)|\leq \epsilon&\\
&\mbox{for any $R$-closed set $U$}&\\
&\forall t\xrightarrow{a} \nu,~\exists t\xrightarrow{a}\mu:~ |\mu(U)-\nu(U)|\leq \epsilon&\\
&\mbox{for any $R$-closed set $U$}&

\end{array}
\right\}
$$

\begin{lemma} $F^{post}_{\epsilon}$ and $F^{pre}_{\epsilon}$ are both monotonic, i.e, $R_1\subseteq R_2$ implies that $F^{post}_{\epsilon}(R_1)\subseteq F^{post}_{\epsilon}(R_2)$ and $F^{pre}_{\epsilon}(R_1)\subseteq F^{pre}_{\epsilon}(R_2)$. Hence, $F^{post}_{\epsilon}$ and $F^{pre}_{\epsilon}$ have the least and the greatest fixed points.
\end{lemma}

\begin{lemma}  Let $\cal {M}=(S,A,\rightarrow)$ be finite branching. Then $F^{post}_{\epsilon}$ and $F^{pre}_{\epsilon}$ are both co-continuous. That is, let $R_{1}\supseteq R_2\supseteq \cdots$, then $F^{post}_{\epsilon}(\cap_{i\in I}R_i)=\cap_{i\in I}F^{post}_{\epsilon}(R_i)$ and $F^{pre}_{\epsilon}(\cap_{i\in I}R_i)=\cap_{i\in I}F^{pre}_{\epsilon}(R_i)$.
\end{lemma}

Let
$$
\begin{array}{lll}
\sim^{0}_{\epsilon}&=& S\times S\\
\sim^{n+1}_{\epsilon}&=&F_{\epsilon}(\sim^{n}_{\epsilon})\quad \mbox{for any $n$}\\
\end{array}
$$
This definition  applies to $\sim^{pre}_{\epsilon}$ and $\sim^{post}_{\epsilon}$.

\begin{proposition} Let $\cal {M}=(S,A,\rightarrow)$ be finite branching. Then $\sim^{pre}_{\epsilon}$ and  $\sim^{post}_{\epsilon}$  are the greatest fixed points of $F^{pre}_{\epsilon}$ and $F^{post}_{\epsilon}$, respectively.
In other words, $F_{\epsilon}=\cap_{i\in N}\sim^{i}_{\epsilon}$ (here $F_{\epsilon}$ refers to $F^{pre}_{\epsilon}$ and $F^{post}_{\epsilon}$).

\end{proposition}

\begin{proof}
Need to prove
\end{proof}

\begin{definition}\label{distance}

(1)
$$
d_{pre}(s,t)=\inf\{\epsilon\mid s\sim^{pre}_{\epsilon} t\}.
$$

(2)
$$
d_{post}(s,t)=\inf\{\epsilon\mid s\sim^{post}_{\epsilon} t\}.
$$

\end{definition}

\begin{proposition}$d_{pre}(s,t)\leq d_{post}(s,t)$. \end{proposition}

\begin{proof} Since $s\sim^{post}_{\epsilon} t$ implies that $s\sim^{pre}_{\epsilon} t$, the set $\{\epsilon\mid s\sim^{post}_{\epsilon} t\}$ is a subset of the set of $\{\epsilon\mid s\sim^{pre}_{\epsilon} t\}$. Hence, $d_{pre}(s,t)\leq d_{post}(s,t)$ holds.
\end{proof}

The following theorem states that the infimum in Definition \ref{distance} of bisimulation
distance can be replaced by minimum, that is, the infimum is achievable.

\begin{theorem} \label {min}Let $\cal {M}=(S,A,\rightarrow)$ be finite branching. Then
$$
\begin{array}{lll}
s&\sim^{post}_{d_{post}(s,t)}& t\\

s&\sim^{pre}_{d_{pre}(s,t)}& t
\end{array}
$$
 \end{theorem}
\begin{proof}
It suffices to prove that
$$
 \begin{aligned}
 R=\{(s,t)\mid s\sim_{\epsilon_{i}} t~&\mbox{for some decreasing sequence}\\
 & \epsilon_1> \epsilon_2>\cdots>0, \mbox{and} \lim_{i\rightarrow \infty}\epsilon_i=\epsilon\}
\end{aligned}
$$
is an $\epsilon$-bisimulation. We choose to prove $s\sim^{post}_{d_{post}(s,t)} t$.

First, the symmetry of $R$ is obvious.


Second, it is easy to see that
$$
R=\bigcap_{k\in N}\sim^{post}_{\epsilon_{k}}
$$
where $\sim^{post}_{\epsilon_{1}}\supseteq \sim^{post}_{\epsilon_{2}}\supseteq \cdots$. Since the state space is finite, there is only finite the relation  $\sim^{post}_{\epsilon_{k}}$. Hence there exists $i_0$ such that
$R=\bigcap_{k\in N}\sim^{post}_{\epsilon_{k}}=\sim^{post}_{\epsilon_{i_0}}=\sim^{post}_{\epsilon_{i_0+1}}\cdots$. That is, $R$ is $\epsilon_{j}(j\geq i_0)$-post-bisimulation. Let $(s,t)\in R$ and $s\xrightarrow{a}\mu$  Hence there exists $t\xrightarrow a\nu_{j}$ ($j\geq i_0$) such that the following holds:
$$
|\mu(U)-\nu_{j}(U)|\leq \epsilon_j
$$
for all $R$-closed sets $U$. Note that $U$ is $\sim^{post}_{\epsilon_{i_0}}$-closed ($j\geq i_0$) too. Finite branching condition implies that a distribution $\nu$ exists such that 
$$
|\mu(U)-\nu(U)|\leq \epsilon_k
$$
for infinite $\epsilon_k$ ($k\geq i_0$). Hence
$$
|\mu(U)-\nu(U)|\leq \inf_{k}\epsilon_k
$$
That is, $|\mu(U)-\nu(U)|\leq  \epsilon$. This shows that $R$ is an $\epsilon$-post-bisimulation. The proof is completed. 
\end{proof}



\begin{proposition} $d(s,t)$ is a pseudometric, that is

(1) $d(s,s)=0$;

(2)$d(s,t)=d(t,s)$;

(3)$d(s,t)\leq d(s,s')+d(s',t)$.

\end{proposition}

\begin{proof} (1) and (2) are obvious, we prove (3).
Let $d(s,s')=\lambda_1$ and $d(s',t)=\lambda_2$. Then for any $\epsilon>0$, there exist $a$ and $b$ such that $s\sim_{a} s'$ and $s'\sim_{b} t$, moreover $a<\lambda_1+\frac{\epsilon}{2}$ and $b<\lambda_2+\frac{\epsilon}{2}$.
Then, the definition of $d$ implies that $s\sim_{\lambda_1+\frac{\epsilon}{2}} s'$ and $s'\sim_{\lambda_2+\frac{\epsilon}{2}} t$. Further, by Proposition \ref{property} (2), we have $s\sim_{\lambda_1+\lambda_2+\epsilon} t$.
Hence, $d(s,t)\leq \lambda_1+\lambda_2+\epsilon$. Since $\epsilon$ is arbitrary, $d(s,t)\leq \lambda_1+\lambda_2$ holds. That is, $d(s,t)\leq d(s,s')+d(s',t)$.


\end{proof}

\begin{proposition} $d(s,t)=0$ iff $s$ and $t$ are bisimilar. \end{proposition}
\begin{proof}
First, by Theorem \ref{min}, $d(s,t)=0$ implies that $s\sim_{0} t$, i.e., $s$ and $t$ are bisimilar. The inverse direction is straightforward.
\end{proof}

\vskip1cm

\R {Question: how to compute $d_{pre}(s,t)$ and $d_{post}(s,t)$}




\end{document}
